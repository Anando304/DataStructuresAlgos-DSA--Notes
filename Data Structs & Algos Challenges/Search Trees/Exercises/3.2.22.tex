\documentclass[11pt,fleqn]{article}

\setlength {\topmargin} {-.15in}
\setlength {\textheight} {8.6in}

\usepackage{amsmath}
\usepackage{amssymb}
\usepackage{color}
\usepackage{tikz}
\usetikzlibrary{automata,positioning,arrows}
\usepackage{diagbox}



\newcommand{\be}{\begin{enumerate}}
\newcommand{\ee}{\end{enumerate}}

\begin{document}
\textbf{Ex 3.2.22:} Prove that if a node in a BST has two children, its successor has no left child and
its predecessor has no right child.\\
	
\textbf{Solution:}\\
For any node, its successor will be leftmost element of its right child. Therefore, this node cannot have a left-child as that is the leftmost-element of its right child.\\

For any node, predecessor will be rightmost element of its left child.

\end{document}
