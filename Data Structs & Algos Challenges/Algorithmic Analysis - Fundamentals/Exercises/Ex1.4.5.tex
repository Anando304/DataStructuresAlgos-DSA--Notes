\documentclass[11pt,fleqn]{article}

\setlength {\topmargin} {-.15in}
\setlength {\textheight} {8.6in}

\usepackage{amsmath}
\usepackage{amssymb}
\usepackage{color}
\usepackage{tikz}
\usetikzlibrary{automata,positioning,arrows}
\usepackage{diagbox}
\usepackage{stackrel}
\begin{document}


1.4.5 Give tilde approximations for the following quantities:\\
a) N + 1 is equivalent to $\sim N$\\
b) 1 + 1/N is equivalent to $\sim 1$\\
c) (1  +  1/N )(1  +  2/N ) is equivalent to $\sim 1$ since 1 is dominant after expansion\\
d) ($2N^3$) – ($15N^2$) + N is equivalent to $\sim2N^3$, since $N^3$ is dominant\\
e) $log(2N)/log(N)$\\
= $\frac{lg(2) + lg(N)}{lg(N)}$\\
= $\frac{log(2)}{log(N)}$ + $\frac{log(N)}{log(N)}$\\
= $\sim 1$\\
f) $\frac{log(N^2+1)}{log(N)}$\\
= $2 \frac{log(N)}{log(N)}$ //$Used \thickspace Logarithm \thickspace power \thickspace law. \thickspace Ignore \thickspace the \thickspace 1 \thickspace since \thickspace log \thickspace is \thickspace dominant$\\
= $\sim 2$
g) $\frac{N^100}{2^N}$\\
= $\sim 0$ since limit approaches 0

\end{document}

