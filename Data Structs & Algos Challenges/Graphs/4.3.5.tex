\documentclass[11pt,fleqn]{article}

\setlength {\topmargin} {-.15in}
\setlength {\textheight} {8.6in}

\usepackage{amsmath}
\usepackage{amssymb}
\usepackage{color}
\usepackage{tikz}
\usetikzlibrary{automata,positioning,arrows}
\usepackage{diagbox}



\newcommand{\be}{\begin{enumerate}}
\newcommand{\ee}{\end{enumerate}}

\begin{document}

\textbf{Exercise 4.3.5:} Show that the greedy algorithm is valid even when edge weights are not distinct.\\

\textbf{Solution:}\\

\begin{center}
	\item Using cut property, having 2 edges of equal weight doesn't negate validity as long as the 2 edges are between 2 different subsets of nodes of the graph. This is so they are isolated in specific semi-graphs without interference and connect together by crossing edge.
	\item we pick an edge and find closest edge, color it black, and do this again for each edge until V-1 times since E=V-1. This basically then simulates Prim's Algorithm.
\end{center}



	


\end{document}
