\documentclass[11pt,fleqn]{article}

\setlength {\topmargin} {-.15in}
\setlength {\textheight} {8.6in}

\usepackage{amsmath}
\usepackage{amssymb}
\usepackage{color}
\usepackage{tikz}
\usetikzlibrary{automata,positioning,arrows}
\usepackage{diagbox}



\newcommand{\be}{\begin{enumerate}}
\newcommand{\ee}{\end{enumerate}}

\begin{document}

\textbf{Exercise 4.2.10:}: What is the maximum number of edges in a digraph with V vertices and no parallel
edges? What is the minimum number of edges in a digraph with V vertices, none of
which are isolated?\\

\textbf{Solution:}\\
Yes, reverse Post-Order DAG(Topological Sort). Basically DFS but push output onto stack instead of print.

\begin{itemize}
	\item There are 3 cases, Proposition F:
	\item dfs(w) has already been called and has returned (w is marked).
	\item dfs(w) has not yet been called (w is unmarked), so $v\rightarrow w$ will cause dfs(w) to
be called (and return), either directly or indirectly, before dfs(v) returns.
	\item dfs(w) has been called and has not yet returned when dfs(v) is called. The
key to the proof is that this case is impossible in a DAG, because the recursive
call chain implies a path from w to v and $v \rightarrow w$ would complete a directed
cycle.
\end{itemize}

The third case has a cycle becayse we see dfs(w) but has not yet been returned, meaning w is still in queue/stack as it was not yet popped/marked, meaning a cycle can occur as not put in marked list yet.

	


\end{document}
