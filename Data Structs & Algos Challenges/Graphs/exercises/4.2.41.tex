\documentclass[11pt,fleqn]{article}

\setlength {\topmargin} {-.15in}
\setlength {\textheight} {8.6in}

\usepackage{amsmath}
\usepackage{amssymb}
\usepackage{color}
\usepackage{tikz}
\usetikzlibrary{automata,positioning,arrows}
\usepackage{diagbox}
\usepackage{stackrel}

\newcommand{\be}{\begin{enumerate}}
\newcommand{\ee}{\end{enumerate}}

\begin{document}
\textbf{Exercise 4.2.41:} Add length directed cycle. Design a linearithmic algorithm to determine whether a digraph has an odd-length directed cycle\\

\textbf{Solution:}
By Anando Zaman


\begin{itemize}
	\item Digraph has odd-length cycle iff one or more of its SCC is nonbipartite.
	
	\item Recall bipartite is a graph whose vertices can be divided into 2 sets such that all edges connect a vertex in 1 set with a vertex in the other.
	
	\item We can run Kosaraju Sharirs SCC algorithm to create a SCC. This algorithm uses DFS once on reverse graph to calculate reverse post-order. It then uses this to determine which nodes to run DFS on for original graph to compute/create the SCC(Strongly connected components).
	
	\item If an edge $v \thickspace \rightarrow \thickspace w$ is pointing wrong direction, we can replace it with an odd-length path that is pointing in opposite direction.
	
	\item If path has odd length, then we replace edge $v \thickspace \rightarrow \thickspace w$ path. If even-length, then path combined with $v \thickspace \rightarrow \thickspace w$ for odd-cycle formation.
\end{itemize}



\end{document}
