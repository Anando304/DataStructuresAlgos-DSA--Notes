\documentclass[11pt,fleqn]{article}

\setlength {\topmargin} {-.15in}
\setlength {\textheight} {8.6in}

\usepackage{amsmath}
\usepackage{amssymb}
\usepackage{color}
\usepackage{tikz}
\usetikzlibrary{automata,positioning,arrows}
\usepackage{diagbox}



\newcommand{\be}{\begin{enumerate}}
\newcommand{\ee}{\end{enumerate}}

\begin{document}

\textbf{Exercise 4.2.1:}: What is the maximum number of edges in a digraph with V vertices and no parallel
edges? What is the minimum number of edges in a digraph with V vertices, none of
which are isolated?\\

\textbf{Solution:}\\
Recall for undirected graph with V vertices, there are $\frac{V(V-1)}{2}$ edges. So 1 edge for both directions per node.\\

For digraph, we have two edges, 1 for each direction. Therefore, there are $\frac{V(V-1)}{2} \thickspace * \thickspace 2 \thickspace \equiv \thickspace V(V-1)$ edges.\\

So $V(V-1)$ max edges while min edge is 1

	


\end{document}
