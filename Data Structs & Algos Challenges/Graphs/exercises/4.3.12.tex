\documentclass[11pt,fleqn]{article}

\setlength {\topmargin} {-.15in}
\setlength {\textheight} {8.6in}

\usepackage{amsmath}
\usepackage{amssymb}
\usepackage{color}
\usepackage{tikz}
\usetikzlibrary{automata,positioning,arrows}
\usepackage{diagbox}



\newcommand{\be}{\begin{enumerate}}
\newcommand{\ee}{\end{enumerate}}

\begin{document}

\textbf{Exercise 4.3.12:} Suppose that a graph has distinct edge weights. Does its shortest edge have to
belong to the MST? Can its longest edge belong to the MST? Does a min-weight edge
on every cycle have to belong to the MST? Prove your answer to each question or give
a counterexample.\\

\textbf{Solution:}
\be
	\item Yes, it must because it is first edge created using Kruskalls as it picks in ascending order for edges.
	
	\item The heaviest edge could belong if it is the only edge connecting a node to rest of the graph.
	
	\item No, the statement 'Min weight edge of a cycle must exist because all edges except longest edge in a cycle exist' is false. Every min weight edge in every cycle DOES NOT have to belong to the MST.
\ee


	


\end{document}
