\documentclass[11pt,fleqn]{article}

\setlength {\topmargin} {-.15in}
\setlength {\textheight} {8.6in}

\usepackage{amsmath}
\usepackage{amssymb}
\usepackage{color}
\usepackage{tikz}
\usetikzlibrary{automata,positioning,arrows}
\usepackage{diagbox}
\usepackage{stackrel}

\newcommand{\be}{\begin{enumerate}}
\newcommand{\ee}{\end{enumerate}}

\begin{document}
\textbf{Exercise 5.3.11 Worst-case Boyer-Moore Algorithm:} Construct a worst-case example for the Boyer-Moore implementation in Algorithm
5.7 (which demonstrates that it is not linear-time).\\

\textbf{Solution:} Recall Boyer-Moore goes from right to left with right side being indices that are multiples of the (length of pattern)-1.\\

Pattern: BAA\\
Text: AAAAAAAAAA\\

This results in worst-case complexity of $O(N*M)$ where N represents size of text and M represents the size of pattern. If for each index in the text, the entire length of pattern is checked every time, then that results in $O(N*M)$ which gives us Quadratic time complexity. $\sim O(N^2)$
\end{document}
