\documentclass[11pt,fleqn]{article}

\setlength {\topmargin} {-.15in}
\setlength {\textheight} {8.6in}

\usepackage{amsmath}
\usepackage{amssymb}
\usepackage{color}
\usepackage{tikz}
\usetikzlibrary{automata,positioning,arrows}
\usepackage{diagbox}
\usepackage{stackrel}

\newcommand{\be}{\begin{enumerate}}
\newcommand{\ee}{\end{enumerate}}

\begin{document}
\textbf{Exercise 5.3.21 Rabin-Karp Algorithm wildcards:} How would you modify the Rabin-Karp algorithm to search for a given pattern
with the additional proviso that the middle character is a “wildcard” (any text character at all can match it).\\

\textbf{Solution:} Recall Rabin-Karp algorithm uses modular hashing.

\begin{itemize}
	\item Since the middle character can be any character, we can skip this character and try comparing with the characters left and right of it.
	
	\item Essentially, we could compute text-hashes skipping middle character as that is where the wildcard is.
	
	\item The algorithm follows same process as before but just skips the middle characters each time when using the text and pattern comparison.
\end{itemize}
\end{document}
