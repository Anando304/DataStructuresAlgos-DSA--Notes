\documentclass[11pt,fleqn]{article}

\setlength {\topmargin} {-.15in}
\setlength {\textheight} {8.6in}

\usepackage{amsmath}
\usepackage{amssymb}
\usepackage{color}
\usepackage{tikz}
\usetikzlibrary{automata,positioning,arrows}
\usepackage{diagbox}



\newcommand{\be}{\begin{enumerate}}
\newcommand{\ee}{\end{enumerate}}

\begin{document}
\textbf{Ex 2.3.20: Nonrecursive quicksort}\\ Implement a nonrecursive version of quicksort based
on a main loop where a subarray is popped from a stack to be partitioned, and the resulting
subarrays are pushed onto the stack. Note : Push the larger of the subarrays onto
the stack first, which guarantees that the stack will have at most lg N entries.\\
	
\textbf{Solution:}
The stack is used to keep track of the 'partition parts' along the path of execution, so its size corresponds to the recursive depth. Since we store atleast half(ie; the larger of the subarrays) of remaining elements of stack every time, this number is halved with every partition which results in a depth of at most $logN$.

\end{document}
