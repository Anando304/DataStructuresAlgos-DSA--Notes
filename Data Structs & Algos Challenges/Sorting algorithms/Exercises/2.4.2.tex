\documentclass[11pt,fleqn]{article}

\setlength {\topmargin} {-.15in}
\setlength {\textheight} {8.6in}

\usepackage{amsmath}
\usepackage{amssymb}
\usepackage{color}
\usepackage{tikz}
\usetikzlibrary{automata,positioning,arrows}
\usepackage{diagbox}



\newcommand{\be}{\begin{enumerate}}
\newcommand{\ee}{\end{enumerate}}

\begin{document}
\textbf{Ex 2.4.2:} Criticize the following idea: To implement find the maximum in constant time,
why not use a stack or a queue, but keep track of the maximum value inserted so far,
then return that value for find the maximum?\\

\textbf{Solution:}\\

\begin{itemize}
	\item Priority Queue acts like a Stack and Queue.
	\item Keeps track of max element with max element at the top.
	\item Typically heaps used to make Priority Queues due to fast $nLogN$ complexity of insert/delMax.
	\item For stacks and queues, we have to go through elements linearly to promote to 'max', so cannot be done in constant time.
	\item Similar to why insert and delMax are not always good complexities depending on sorted or unsorted arrays.
	
\end{itemize}

\end{document}
